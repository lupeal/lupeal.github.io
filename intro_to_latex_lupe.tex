\documentclass[11pt]{article}
\usepackage{graphicx} % Required for inserting images
\usepackage[margin=1in]{geometry}
\usepackage{amsmath}
\usepackage{graphicx}
\usepackage{float}
\usepackage{url}

\title{Intro to \LaTeX}
\author{Lupe Antonio Lopez}
\date{November 15, 2023}

\begin{document}

\maketitle

\section{What is \LaTeX?}

\LaTeX is a document preparation system used to create professional-looking documents. It complies unformatted text and commands it into a prepared document without having to worry too much about the layout or appearance of the content. \LaTeX is especially useful when writing complicated mathematical equations for homeworks, projects, reports, etc. 

\noindent You could say, it's like a programming language. It expects a particular syntax, or else it won't work. 

\noindent You may be asking yourself, why should I use \LaTeX? Totally, valid question! Personally, \LaTeX is appealing because it allows for precise and concise presentations of mathematical ideas. You are able to implement professional-looking tables, figures, and equations to help aid your presentations and research! 

Now, let's dive into some basics!

\section{Where to use \LaTeX}

\begin{itemize}
    \item Overleaf is an easy-to-use online \LaTeX editor, comparable to Google Docs. This is my preferred method when creating documents or homework assignments that don't involved a lot of R code. Note: use your Columbia email to access a premium account! You add unlimited collaborators to a project. 

    \item RMarkdown allows for a seamless incorporation of \LaTeX. In order to use \LaTeX in an .RMD file, you must add \$ before any \LaTeX language.

    \item There are editor desktop apps like Texmaker and TexShop, comparable to Word. Note that these do not require internet and are great for on-the-go. 
\end{itemize}

Regardless, they all use the same \LaTeX syntax!


\section{Starting a document in Overleaf}

Create a New Project to access a \verb|.tex| file, where you will compile the source file with text and/or \LaTeX. To your right, the compiled \verb|.pdf| file will appear.


\noindent Now, some layout/aesthetic basics:

\begin{enumerate}
    \item \verb|\documentclass|: general format of the document (\verb|article|, \verb|book|, etc)

    \item Font size: 10pt, 11pt, 12pt

    \item \verb|\usepackage{}|: lots of packages available on Overleaf (no installation needed)

    \item \verb|\begin{document}...\end{document}|

    \item All commands start with a backslash \verb|\|
\end{enumerate}


\section{Mathematical Environments}

\subsection{In-line}
In-line \LaTeX is created when surrounding syntax with only one \$ on either side. This is similar to in-line R code.

\noindent An example of in-line \LaTeX:
The \verb|$\sqrt{25} + \sqrt{9}$| is 8.

\noindent This render to:
The $\sqrt{25} + \sqrt{9}$ is 8. 


\subsection{Single-line equations}
Sometimes you'd like to showcase an important result or a longer mathematical expression. To do so, you enclose the mathematical object with \verb|\begin{equation}| and \verb|\end{equation}|. As an example, let's look at the PMF of a Poisson random variable,

\begin{equation}
    f(x) = \frac{\lambda^{x}e^{-\lambda}}{x!}
\end{equation}


\subsection{Multi-line equations}
A natural extension to single-line equations is a multi-line equations. This is useful for proofs or presenting all your steps to your audience. Here's an output example,
\begin{align*}
x \in (A \cup B)^{c} &= x \notin A \cup B \tag{step 1} \\
                     &= x \notin A \land x \notin B \tag{step 2} \\
                     &= x \in A^{c} \land x \in B^{c} \tag{step 3} \\
                     &= x \in A^{c} \cap B^{c} \tag{step 4}
\end{align*}

To create multi-line equations, enclose your objects with commands \verb|\begin{align}...\end{align}| or \verb|\begin{align*}...\end{align*}| to suppress number tags and/or add your own tags. To add your own tag to any line, use command \verb|\tag{tag here}|. To align your equations, add \textbf{\&} to create vertical line breaks. To end a line, use \verb|\\| to create horizontal line breaks.

\section{Structure}

\subsection{Sections}
\begin{verbatim}
    \section
    \subsection
    \subsubsection
\end{verbatim}

\subsection{Lists}

\begin{itemize}
    \item Add bullet points using the commands \verb|\begin{itemize}...\end{itemize}|
\end{itemize}

\begin{enumerate}
    \item Add number lists using the commands \verb|\begin{enumerate}...\end{enumerate}|
\end{enumerate}

Note: each item in the lists are identified by the \verb|\item| command. 


\section{Useful Expressions \& Commands}
Remember to add \$ before any \LaTeX syntax or commands. 
\begin{itemize}
    \item[] \verb|\frac{x}{y}| = $\frac{x}{y}$
    \item[] \verb|e^{x}| = $e^{x}$
    \item[] \verb|X_{i + j}| = $X_{i + j}$
    \item[] \verb|\bar{X}| = $\bar{X}$
    \item[] \verb|\mu| = $\mu$
    \item[] \verb|\hat{\sigma}| = $\hat{\sigma}$
    \item[] \verb|\sum| = $\sum$
    \item[] \verb|\prod| = $\prod$
    \item[] \verb|\in| = $\in$
    \item[] \verb|\infty| = $\infty$
    \item[] \verb|\int| = $\int$ 
    \item[] \verb|X_1, \ldots, X_n| = $X_1, \ldots, X_n$
    \item[] \verb|\{| \verb|\}| = $\{ \}$
    (Add a backslash before a symbol if you'd like for it to appear in text or an equation.)
\end{itemize}



\section{Tables}
There are now two different ways to create a table with \LaTeX. 

\begin{table}[h]
    \centering
    \begin{tabular}{|c|c|} \hline
    yes & no \\ \hline
    2   & 10 \\ \hline
    5   & 50 \\ \hline
    \end{tabular}
    \caption{Example Table}
    \label{tab:my_label}
\end{table}


\begin{itemize}
    \item[] One way is using the commands \verb|\begin{table}[h]...\end{table}|. In between the two commands and after the command \verb|\begin{tabular}|, you will see some preset lines with \& signs. On either side of the \& sign, you can add values or words to fill in the table. Include \verb|\hline| before and after lines to enclose values into a table. Also, include a label to refer back to it within your text using \verb|\ref{tab:name of table}|. 

    \item[] An easier way to create a table in Overleaf is using the Visual Editor. Once in Visual Editor, click on the three dots (...) to access the \textbf{Insert Table} option. 
\end{itemize}


\section{Figures}
\LaTeX also allows for the incorporation of figures and images into your document. In Overleaf, upload your image into the left-hand side. Using the commands \verb|\begin{figure}...\end{figure}|, include the file name in command \verb|\includegraphics{file_name_here}|. Similar to tables, you can label your figures to refer back to it within your text using \verb|\ref{fig:name of image}|. Here is an output example,

\begin{figure}[H]
    \centering
    \includegraphics[scale=0.4]{nyc_skyline_columbia.jpeg}
    \caption{NYC Skyline}
    \label{fig:Columbia_skyline}
\end{figure}


\section{Extra Resources}

\begin{itemize}
    \item Here is a website that has compiled various sources related to \LaTeX: \url{https://texblog.org/tex-resources/}

    \item If you'd like more information on citations for papers/reports: \url{https://libguides.eur.nl/overleaf/bibliographies-and-citing#:~:text=Citing%20in%20LaTeX,is%20also%20a%20popular%20option.}

    \item To get started on Overleaf, the basefiles and templates provided are a great start!   
\end{itemize}

\end{document}
